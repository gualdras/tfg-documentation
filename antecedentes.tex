\chapter{Antecedentes}
\label{chap:antecedentes}

En este capítulo se discutirán algunos de los conceptos y tecnologías que se han empleado para la elaboración de este proyecto. Por una parte se hablará sobre los sistemas de recomendación, explicando en que consisten y profundizando especialmente en los sistemas de recomendación híbridos. Por otro lado se explicarán algunas de las tecnologías empleadas. %Respecto a las tecnologías no se si hablar por ejemplo del google app engine o por el contrario de la computación en la nube en general y decir que en este caso he usado el sistema de google. (cse, google messaging) 

\section{Sistemas de recomendación}
En la vida diaria, las personas recurren a las recomendaciones o críticas por parte de otras cuando sus experiencias personales o conocimientos sobre el tema son insuficientes.  Los sistemas de recomendación actuales desempeñan un papel similar a este proceso. En 1997 Resnick y Varian \cite{Resnick:1997:RS:245108.245121} definieron por primera vez los sistemas de recomendación como sistemas en los que "la gente proporciona recomendaciones como entrada al sistema, el cual luego se encarga de agregar y redirigir estas a los destinatarios adecuados". %no se si tengo que citar entre comillas cuando hago mi traducción no oficial
Posteriormente se ha ampliado esta definición, considerando sistemas de recomendación aquellos que generan recomendaciones personalizadas o que son capaces de guiar al usuario hacia elementos que sean de su interés dentro de un amplio abanico de posibilidades. Es esta característica de individualización o recomendación personalizada lo que distingue a estos sistemas de otros como los buscadores simples y los sistemas de recuperación de información.

Estos sistemas son claves hoy en día donde toda la información que se encuentra en la red en prácticamente cualquier portal, trasciende la capacidad de los usuarios para buscar y seleccionar de entre todos los elementos por si mismo. Un ejemplo de sistema de recomendación es el sistema de la compañía Netflix, que proporciona recomendaciones de películas en base a las anteriores visualizaciones de los usuarios. Esta compañía realizó en 2006 una competición en la que retó a la comunidad a desarrollar un sistema de recomendación que fuese capaz de derrotar al de la compañía \textit{Cinematch}%aquí no se si citar la fuente que aparece en el archivo que me pasastes de lu recommender systems. Porque cuando habla de este caso pone una fuente externa J. Bennett, S. Lanning, The Netflix prize, in: Proceedings of KDD Cup and Workshop, 2007, pp. 3–6.
. Para ello ofreció al público una pequeña fracción de sus datos en la que se encontraban valoraciones anónimas de usuarios sobre películas. Se inscribieron 20.000 equipos, de los cuales 2.000 presentaron al menos una solución. En 2009 se concedió un premio de 1.000.000 de dolares a un equipo que mejoró la precisión de Cinematch en un 10\%. Todos estos números pueden servir para hacerse una idea del potencial que ofrecen los sistemas de recomendación hoy en día.

\subsection{Clasificación de los sistemas de recomendación}

Desde su aparición se han realizado distintas clasificaciones de los sistemas de recomendación. En este caso se va a mostrar la clasificación que realizaron Resnick y Varian \cite{Resnick:1997:RS:245108.245121}. De este modo, se establecen una serie de características según las cuales se podría realizar la clasificación de estos sistemas. En primer lugar se establecen las siguientes características técnicas:

\begin{itemize}
\item El contenido de la recomendación, es decir la valoración a los distintos elementos a recomendar. Esta valoración puede ser tan simple como un bit (recomendado o no) o algo más complejo como un un porcentaje o un texto.
\item El modo en el que se realizan o recogen las recomendaciones. Las recomendaciones pueden ser realizadas de manera explícita, pero también pueden ser recogidas de manera implícita por el sistema, por ejemplo analizando las preferencias de los usuarios, las búsquedas y visitas de estos a diferentes portales o las compras previas de estos. Existen mecanismos muy útiles para poder recoger estas recomendaciones implícitasm como las cookies en los sistemas web.
\item La identidad de los recomendadores. Esta puede ser la identidad real, un pseudónimo o bien una identidad anónima. Como se verá más adelante, los usuarios prefieren conservar su privacidad en numerosas ocasiones, y pueden ser reacios a compartir información de carácter sensible aun cuando esta información puede ser crucial para el sistema de recomendación, por esta razón es necesario ofrecer mecanismos que les permitan conservar dicha privacidad.
\item Las técnicas de recomendación, es decir, la manera en la que se relacionan las recomendaciones con aquellos que buscan recomendación. Esta es una de las características que permiten mayor flexibilidad en este tipo de sistemas y será analizada en profundidad más adelante.
\item La finalidad de las evaluaciones. Uno de los usos es el de descartar o sugerir elementos. Otra finalidad puede ser la de ordenar los elementos recomendados según un peso, o mostrar para cada elemento el nivel de recomendación. 
\end{itemize}

Además de las características técnicas, otro de los elementos que caracterizan un sistema de recomendación es su dominio, es decir, los elementos o items que se recomiendan, y el público que realiza o recibe las distintas recomendaciones.

En lo referente a los elementos sobre los que se aplican las recomendaciones es importante, en primer lugar, definir el tipo de los elementos que se están recomendando. En el Cuadro~\ref{tab:items-recommended} se muestran algunos ejemplos de sitios y los elementos que recomiendan. Otro factor importante es el volumen de los elementos que se recomiendan, así como la frecuencia con la que se generan y desaparecen. Es necesario conocer y tener en cuenta estos parámetros ya que se deben de tratar de manera muy distinta unos elementos que se generan con gran frecuencia y tienen un tiempo de vida corto, como pueden ser las noticias de un medio electrónico en el que es muy importante poder recomendar dichas noticias en un tiempo acotado, a la necesidad de recomendar por ejemplo películas o libros. Es aquí donde asumen gran importancia las técnicas de recomendación. Por otro lado, se encuentran los costes que implican las diferentes posibilidades en cuanto a la recomendación. Es posible que el coste de fallar en la recomendación de un buen o mal ítem sea alto, o por otro lado el coste de un análisis intensivo puede ser mayor. Esta característica es altamente dependiente de la configuración de las características técnicas mencionadas anteriormente.  

\begin{table}[hp]
  \centering
  {\small
  


\begin{tabular}{p{.3\textwidth}p{.3\textwidth}}
  \tabheadformat
  \tabhead{Sitio}   &
  \tabhead{Elementos recomendado}\\
\hline
Amazon  & Libros/otros productos \\
\hline
Facebook & Amigos\\
\hline
WeFollow  & Amigos \\
\hline
MovieLnes  & Películas\\
\hline
Nanocrowd  & Películas \\
\hline
Jinni  & Películas\\
\hline
Findory  &  Noticias\\
\hline
Digg  &  Noticias\\
\hline
Zite  &  Noticias\\
\hline
Meehive  &  Noticias\\
\hline
Netflix  & DVDs\\
\hline
CDNOW  &  CDs/DVDs\\
\hline
eHarmony  & Citas\\
\hline
Chemistry  &  Citas\\
\hline
True.com  & Citas\\
\hline
Perfectmatch  &  Citas\\
\hline
CareerBuilder  & Trabajos\\
\hline
Monster  &  Trabajos\\
\hline
Pandora  & Música\\
\hline
Muffin  & Música \\
\hline
StumbleUpon  & Páginas Web\\
\hline
\end{tabular}


% Local variables:
%   coding: utf-8
%   ispell-local-dictionary: "castellano8"
%   TeX-master: "main.tex"
% End:

  }
  \caption[Sitios web y elementos que recomiendan]
  {Sitios web y elementos que recomiendan
    (\textsc{RESNICK}~\cite{Lu})}
  \label{tab:items-recommended}
\end{table}


En el caso de los usuarios involucrados en el proceso de las recomendaciones, tanto aquellos que las realizan como los que las consumen, es necesario conocer los perfiles de estos. Por ejemplo se debe saber si los usuarios tienden a realizar recomendaciones de numerosos elementos similares, o por el contrario evalúan solo elementos muy específicos, dando lugar a diferentes conjuntos de recomendaciones. También es importante conocer la cantidad de usuarios que componen o compondrían el sistema, y la variedad respecto a los gustos de los usuarios, por ejemplo, existiendo una gran cantidad de usuarios con gustos similares.



\subsection{Técnicas de recomendación}
Como se ha mencionado anteriormente existen varias técnicas de recomendación que permiten adaptarse a la situación en cuestión. A la hora de elegir una de estas técnicas, es necesario tener en cuenta principalmente tres factores. Por un lado se encuentran los datos y la información de la que disponemos antes de comenzar el proceso de recomendación, como pueden ser las valoraciones de ciertos usuarios de los items, o información sobre dichos items. Por otro lado está la entrada que realiza el usuario,  es decir la información que este comunica al sistema con el propósito de obtener una recomendación. Finalmente es necesario disponer mecanismo o algoritmo que sea capaz de combinar los dos elementos anteriores para poder llevar a cabo la recomendación. Mediante estos tres factores se pueden definir un total de cinco técnicas de recomendación. Considerando \textbf{U} como el conjunto de los usuarios de los que se conocen las preferencias, \textbf{I} el conjunto de los items sobre los que existen valoraciones, \textbf{u} el usuario para el que realizar la recomendación e \textbf{i} el elemento en cuestión que se esta considerando para ser recomendado, en el cuadro \ref{tab:recommendation-techniques} se muestra a grandes rasgos como funcionan estas técnicas en función de los factores anteriores. 


\begin{table}[hp]
  \centering
  {\small
  


\begin{tabular}{p{.15\textwidth}p{.25\textwidth}p{.25\textwidth}p{.25\textwidth}}
  \tabheadformat
  \tabhead{Técnica}   &
  \tabhead{Información previa}      &
  \tabhead{Entrada} &
  \tabhead{Algoritmo}  \\
\hline
Colaborativo & Evaluaciones de \textbf{U} de los elementos en \textbf{I}. & Evaluación de \textbf{u} de los elementos en \textbf{I}. & Identifica a los usuarios en \textbf{U} con perfiles similares a \textbf{u}, y extrapola sus valoraciones de \textbf{i}.\\
\hline
Basada en contenido  & Características de los elementos en \textbf{I}. & Las valoraciones de \textbf{u} de los elementos en \textbf{I}. & Generar un clasificador que adapte el comportamiento respecto a las valoracioes de \textbf{u} y lo use en \textbf{i}. \\
\hline
Demográfica & Información demográfica sobre \textbf{U} y sus evaluaciones de los elementos en \textbf{I}. & Información demográfica sobre \textbf{u}. & Identificar aquellos usuarios que son demográficamente similares a \textbf{u} y extrapolar sus evaluaciones de \textbf{i}. \\
\hline
Basada en utilidad & Características de los elementos en \textbf{I}. & Una función de utilidad sobre los elementos en \textbf{I} que describa las preferencias de \textbf{u}. & Aplicar la función sobre los elementos y determinar la clasificación de \textbf{i}. \\
\hline
Basada en conocimiento & Características de los elementos en \textbf{I}. Conocimiento de como estos elementos cumplen las necesidades del usuario. & Una descripción de las necesidades o intereses de \textbf{u} & Inferir la afinidad del item \textbf{i} con las necesidades de \textbf{u}.\\
\hline
\end{tabular}


% Local variables:
%   coding: utf-8
%   ispell-local-dictionary: "castellano8"
%   TeX-master: "main.tex"
% End:

  }
  \caption[Técnicas de recomendación]
  {Técnicas de recomendación
    (\textsc{BURKE}~\cite{Burke})}
  \label{tab:recommendation-techniques}
\end{table}

A continuación se muestran en más detalle las diferentes técnicas.

\subsubsection{Colavorativa}
La técnica colaborativa es probablemente la más usada dentro de este tipo de sistemas. En los sistemas que usan esta técnica, cada usuario tiene generalmente un perfil en el que se encuentran las evaluaciones que ha realizado este de cada elemento. El funcionamiento de esta técnica consistiría en comparar al usuario, o en este caso su perfil, con el de otros para encontrar las similitudes entre ellos, y usar estas similitudes para extrapolar la información sobre las evaluaciones u opiniones de estos usuarios sobre el item que se pretende evaluar o recomendar. Un ejemplo simple de como sería un perfil de usuario en estos sistemas consiste en una estructura de datos en el que se encuentran los elementos evaluados junto con las valoraciones realizadas por el usuario. Los sistemas que emplean esta técnica están pensados para generar relaciones a largo plazo, ya que por norma general mejoran su funcionamiento conforme aumentan las recomendaciones. Es por esto, que en algunos casos es recomendable establecer mecanismos que sean capaces de modificar estos datos en función del tiempo de manera ajena al usuario. Esto es debido a que el usuario puede haber variado sus gustos u opiniones a lo largo de este tiempo, y las evaluaciones pueden no ser fiables. 


\subsubsection{Demográfica}
En este tipo de recomendadores, es necesario establecer una serie de perfiles o clases definidos de manera previa. Posteriormente se obtiene la información personal de los usuarios y se usa esta para categorizarlos en alguna de las clases demográficas definidas anteriormente. Los mecanismos para obtener estos datos pueden ser variados, usando mecanismos para obtenerlos de manera explícita como puede ser un test, u otros mecanismos que extraigan esta información de manera implícita. Esta técnica es similar a la colavorativa en el uso de los perfiles de los usuarios para sus recomendaciones, sin embargo la ventaja de esta sobre la anterior es que no necesita evaluaciones previas de los items por parte de los usuarios.

\subsubsection{Basada en contenido}
Esta técnica es una extensión de los estudios en el filtrado de información. A diferencia de la técnica colaborativa en la que se establecían relaciones entre los usuarios, en esta se establecen relaciones entre los los elementos a recomendar. Se comparan aquellas características que se encuentran en los diferentes elementos estableciendo el grado de similitud en los valores de estas. Usando esta información es posible sugerir aquellos elementos que tienen características en común con aquellos que el usuario ha valorado positivamente. De esta manera, el perfil de usuario se establece basándose en aquellos elementos que han gustado, o dicho de otra manera que el usuario ha valorado positivamente. Esta técnica puede ser muy útil en un sistema de recomendación de películas por ejemplo, donde encontramos elementos que comparten las mismas características como el reparto, la temática o la fecha de lanzamiento. Al igual que los sistemas colaborativos, estos también mejoran su rendimiento con el aumento de las valoraciones de los usuarios.

\subsubsection{Basada en utilidad}
Los sistemas que emplean esta técnica realizan recomendaciones a partir de un cálculo sobre la utilidad que tiene cada elemento para satisfacer las necesidades del usuario. Es necesario por tanto establecer una función de utilidad para cada usuario, siendo esta función la que componga su perfil. Este tipo de técnica presenta la ventaja de poder incluir en sus cálculos y recomendaciones atributos que no son propios del elemento que se quiere recomendar como tal, como puede ser por ejemplo la confianza en el vendedor de dicho producto en casos de comercio en línea, o el tiempo de entrega. Esto permite que estos sistemas sean capaces de valorar y negociar de acuerdo a distintas características, por ejemplo negociando el precio en función del tiempo de entrega.

\subsubsection{Basada en conocimiento}
Esta técnica realiza las recomendaciones a parir del conocimiento de como un determinado item satisface las necesidades o deseos de un usuario. El perfil de usuario en este tipo de reomendadores será cualquier estructura que sea capaz de representar las necesidades de los usuarios. Del mismo modo, el conocimiento sobre los elementos podrá ser representado y almacenado de diversas formas. Al igual que los recomendadores basados en utilidad, los basados en conocimiento no están diseñados para mejorar su funcionamiento con la adición de usuarios.

\subsection{Dilemas de los sistemas de recomendación}
Los sistemas de recomendación presentan principalmente dos tipos de problemas, uno que se deriva de la técnica utilizada y que va ligado a esta debido a su funcionamiento y por tanto que tiene carácter técnico, y otro tipo de problema que depende en mayor medida del dominio, es decir de los elementos que se recomiendan y los usuarios que hacen uso del sistema.

\subsubsection{Problemas del dominio}
Uno de los problemas en los que los usuarios tienen un papel protagonista es la privacidad, que aunque no es exclusivo de estos sistemas adquiere gran importancia. En estos sistemas suele ser común que la información más valiosa sea también aquella que el usuario es más reacio a compartir. En algunos sistemas como hemos visto anteriormente, es posible la participación en las recomendaciones de manera anónima o bajo un pseudónimo, sin embargo en muchas ocasiones esto puede no ser posible cuando por ejemplo los usuarios quieren reconocimiento al realizr una recomendación, pero no dar a conocer todos los detalles de su información personal. Por esto es necesario en algunos sistemas el proporcionar mecanismos que presenten un grado intermedio de privacidad, ajustándose a las necesidades y deseos de los usuarios. Otro problema que es común a todas las técnicas es el de la creación del perfil de usuario. Muchos usuarios al principio pueden no ver el potencial que puede llegar a tener este sistema y por tanto considerar la inversión de tiempo necesaria para formar su perfil demasiado alta respecto con el beneficio inmediato que le puede aportar. Es por esto que se requiere que los sistemas presenten incentivos para que estos usuarios contribuyan en la creación de los perfiles, o que por otro lado, estos sistemas de recomendación sean capaces de recopilar estas preferencias o necesidades de los usuarios sin necesidad de intervención explicita de estos. Esto es más fácil en algunas técnicas que en otras.

Cuando las recomendaciones dependen de las opiniones de los usuarios como en el caso de las técnicas colaborativas, surge otro problema conocido como "vote early and often" o votar pronto y a menudo en español. Este fenómeno consiste en manipular de alguna manera los votos o en este caso las recomendaciones. En el caso de que cualquiera pueda realizar todas las recomendaciones que desee, los creadores de contenido o proveedores de distintos productos o servicios recurrirán a votar en favor de sus elementos y en detrimento de sus competidores.

\subsubsection{Comparaciónusuarios de técnicas}
Existen problemas van ligados a la técnica escogida, y todas las técnicas presentan una serie de ventajas y desventajas entre si.

Uno de los problemas más conocidos es el problema del comienzo frío, o cuesta arriba. %No se si aqui debería añadir los nombre en ingles, vamos en general cuando hablo de términos que son más bien de origen ingles
Este problema se puede dar en dos casos, cuando aparece un nuevo usuario y cuando aparece un nuevo item. Cuando aparece un nuevo usuario no se tiene información de el, y en el caso de ser necesario relacionarlo con algún otro usuario o con algún perfil concreto resulta complicado. El caso de un nuevo item es similar. Cuando aparece un nuevo item y apenas existen recomendaciones sobre este, aparecen problemas a la hora de relacionar este nuevo item con otros. En ambos casos, si el sistema requiere de recomendaciones, es necesario que este presente algún tipo de incentivo para que los usuarios realicen valoraciones sobre los elementos.

En el caso de los sistemas colaborativos estos se ven afectados por los problemas tanto de nuevo usuario como de nuevo item. Debido a sus características, este sistema presenta problemas en aquellos entornos en los que existen pocos usuarios y estos evalúan continuamente los mismos items (problema conocido como dispersión), o bien los usuarios tienen gustos muy diferentes. En definitiva, este tipo de sistemas resulta útil cuando existe un número suficientemente amplio de usuarios y con una variedad de gustos aceptable dentro del dominio del problema. Una de las ventajas que presentan estos sistemas respecto a otros es principalmente su capacidad para poder recomendar elementos que pese a que no estén dentro del mismo grupo, pueden estar relacionados de alguna manera. Por ejemplo puede darse la situación de que a todos aquellos a los que les gusta la música Jazz les guste también el mismo grupo de rock, y este sistema sería capaz de recomendar dicho grupo, mientras que otros como por ejemplo el basado en contenido no podría. Otra gran ventaja de esta técnica, es la de ser independiente de la representación a nivel de computador de los elementos que recomienda. Debido a esto es frecuentemente usado en la recomendación de elementos u objetos complejos como libros, música o películas.

Los sistemas basados en contenido sufren de manera menor del problema de la dispersión debido a que solo tienen en cuenta las valoraciones del propio usuario y no de todos los usuarios. A pesar de esto, tienen el mismo problema de comienzo en frío que tenían los sistemas colaborativos. Además estos sistemas, a diferencia de los colaborativos están limitados por las características o los datos de los elementos que se recomiendan. Otro problema que tiene en común con los colaborativos surge como consecuencia de que estos sistemas generalmente no deben recomendar un objeto que el usuario ya ha valorado, y que por tanto se entiende que dispone de el o lo conoce lo suficiente. Aunque esta condición no es siempre necesaria, ya que el puede ser útil recomendar algo que el usuario ya haya seleccionado anteriormente, esta opción puede no ser deseada. El problema que se presenta aquí es el de distinguir cuando dos elementos son lo suficientemente parecidos para ser considerados iguales y no ser recomendados al mismo usuario tras haberlo hecho con uno de ellos.

Los sistemas demográficos también sufren el problema del comienzo frío, pero solo respecto a la aparición de nuevos items. También comparten otro de los problemas de los colaborativos que es el de identificar a usuarios que se salen de lo común, es decir, que o bien no tienen características en común con ninguno de los actuales grupos, o presentan algunas características de cada grupo pero no las suficientes como para categorizarlos en uno concreto. Pese a no tener el problema de comienzo frío con los nuevos usuarios, necesitan recopilar información sobre los distintos perfiles antes de comenzar a operar. Esta recopilación y formación de perfiles es su principal diferencia respecto a los sistemas colaborativos.

Pese a que estos sistemas mencionados anteriormente, que están basados en el aprendizaje durante la ejecución, parecen tener muchos problemas especialmente al inició, presentan la ventaja de ser más flexibles y adaptativos que las dos técnicas que se verán a continuación. Tanto los sistemas basados en utilidad como los basados en conocimiento carecen de los problemas de comienzo frío y de dispersión ya que no aprenden durante su ejecución. Sin embargo en ambos casos es necesario un gran trabajo previo.

En el caso de los sistemas basados en utilidad, pese a carecer del problema del comienzo frío necesitan crear la función de utilidad considerando todas las características del objeto. Esto supone un gran trabajo al inicio y mucha interacción con el usuario, lo cual supone una desventaja para usuarios inexpertos o usuarios que no quieren invertir demasiado tiempo en esto, pero que facilita en gran medida el filtro y selección de aquellos usuarios expertos o que buscan elementos con unas características muy específicas. Como se mencionó en la sección anterior, la gran ventaja de estos sistemas es poder considerar aquellas propiedades que no son intrínsecas al objeto, como el tiempo de el formato del envío. 

Por otra parte, los sistemas basados en conocimiento comparten el problema de la recolección de datos o conocimiento con los sistemas clásicos basados en el conocimiento, en los que se necesita de un experto del que extraer información y de un ingeniero del conocimiento que sea capaz de extraer dicha información. Respecto a los tipos de conocimiento que estos sistemas requieren se pueden categorizar en tres. Por una parte encontramos el conocimiento sobre los items o elementos que se van a recomendar. También encontramos el conocimiento sobre el usuario y las necesidades de este. Finalmente está el conocimiento funcional, que permite establecer relación entre las necesidades del usuario y las características de los objetos.
%En general no se si tengo suficientes citas, ya que solo he puesto las que me he leído, no se si en alguna parte o en algun caso sería recomendable poner alguna más que tu consideres

\subsection{Sistemas de recomendación híbridos}

En la sección anterior se ha visto que cada tipo de sistema de recomendación tiene unas fortalezas y debilidades, las cuales en muchos casos son complementarias. Es decir, algunos sistemas poseen aquello que les falta a otros y viceversa. Siendo esto así surge una idea, la de combinar varios de estos sistemas tratando de buscar sacar el máximo potencial posible de cada uno.


% Local Variables:
%  coding: utf-8
%  mode: latex
%  mode: flyspell
%  ispell-local-dictionary: "castellano8"
% End:
