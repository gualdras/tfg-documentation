\chapter{Resumen}

Uno de los objetivos de la informática es facilitar la realización de ciertas actividades y tareas a los usuarios, especialmente a aquellos que por una razón u otra tienen mayores dificultades a la hora de desempeñar alguna de estas labores. Hay muchas personas, que por razones como la edad, algún tipo de enfermedad, alguna minusvalía y/o discapacidad, tienen problemas a la hora de leer y escribir, especialmente utilizando los medios electrónicos actuales. El problema se acentúa notablemente en los dispositivos móviles smartphones, donde la pantalla es de un tamaño reducido, así como el teclado disponible para escribir. Este problema de lectura y escritura se traslada también al terreno de las telecomunicaciones. En un mundo en el que la mayor parte de la comunicación electrónica se realiza de manera escrita (obviando las llamadas), ya sea mediante las aplicaciones de mensajería que se utilizan en dispositivos móviles, como las redes sociales, foros o comentarios, estas personas quedan aisladas y apartadas.  Este Trabajo Fin de Grado pretende ayudar a estas personas en el ámbito de la comunicación a través de Internet.


Se propone con este fin el desarrollo de una aplicación de mensajería para la plataforma Android. Esta aplicación, a partir de una entrada que bien sea escrita, pero que en la mayoría de las ocasiones se realizará mediante reconocimiento de voz, propondrá al usuario una serie de imágenes. Se usarán tanto pictogramas creados especialmente para la comunicación de personas que tienen problemas en este área, como imágenes o fotografías de propósito y carácter general que se encuentren en una serie de sitios web que se seleccionarán para este propósito. Estas imágenes, se le presentarán o propondrán al sujeto haciendo uso de un sistema de recomendación, el cual, a partir de la entrada proporcionada por el usuario que para el que se está realizando la recomendación, las elecciones previas realizadas por este, así como por las elecciones de otros usuarios, recomendará una serie de imágenes.


\chapter{Abstract}

One of the main goals of computer science is to help people in the development of their activities and tasks. This applies particularly to those people that for whatever reason have problems in the performance of those tasks. There are a lot of people that for reasons such as the age, some illnesses, or some handicap and/or disability, have problems with activities of reading and writing, especially when they have to use the electronic means. This problem is even bigger when using mobile devices such as smartphones, where there is a very small screen and keyboard. The problem of writing and reading is also present in the field of telecommunications. In a world where most of the on-line communication is performed written (except for the phone calls), either by messaging applications, social networks, forum or comments in websites, this people are being left out or neglected. That is the purpose of this project,  which aims to help this people in the communication throw the Internet.

It is proposed for this purpose the development of a messaging application for the Android platform. This application, starting with the input provided by the user that will be either written, or more probably by voice, will provided the user with a set of images. The images used will be both pictograms especially made for the communication of people with problems in this area, and images or photographs of general purpose located in some web sites selected for this reason. This images will be proposed  to the user through a recommender system. This system will use the information available from previous choices of images by the user, and the choices of other users to provide a recommendation.