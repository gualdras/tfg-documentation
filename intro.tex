\chapter{Introducción}
\label{chap:intro}

\noindent
\drop{E}{}xiste una gran cantidad de personas que tienen dificultades para comunicarse a través de los medios de comunicación escritos. Las populares aplicaciones de mensajería que se comercializan hoy día, utilizan mayoritariamente este tipo de comunicación, y aunque  aportan la posibilidad de enviar otro tipo de archivos o información, no facilitan esta tarea. Se propone para ayudar a estas personas, el desarrollo de una aplicación de mensajería que, a diferencia de las ya actuales, permita a los usuarios utilizar imágenes como principal mecanismo de comunicación, haciendo uso para esto de un sistema de recomendación. Además, este sistema puede ser muy útil para todo tipo de personas, independientemente de si tienen problemas con la comunicación escrita o no, ya que aporta una funcionalidad que muchos usuarios encontrarán útil durante sus conversaciones.


Los sistemas de recomendación están cada vez más presentes en el mundo de la informática. Una gran cantidad de sitios web tienen una cantidad de artículos o elementos superior a la que los usuarios que acceden a sus servicios pueden consultar. Este tipo de sistemas se utilizan en sitios desde portales de venta online como amazon \footnote{www.amazon.es}, en redes sociales para recomendar contactos como facebook \footnote{www.facebook.com} o en sitios de distribución de contenido multimedia como netflix \footnote{https://www.netflix.com/}. 

En este caso, el contenido de la recomendación de este sistema serán imágenes. La razón para haber seleccionado un sistema de este tipo, y no haber recurrido simplemente a un sistema de filtrado o recuperación de información o simplemente un buscador, es debido a que además de presentar una ventaja notoria frente a estos sistemas una vez se tiene suficiente información sobre el usuario, son especialmente adecuados en aplicaciones en las que existen usuarios con perfiles y gustos comunes. Debido a que está orientado a personas que generalmente van a tener una serie de rasgos en común, como por ejemplo personas con síndrome de Down, será más fácil identificar a estas personas dentro de un grupo de usuarios concreto con mayor facilidad que si se tratase de otro tipo de usuarios. Así, si varios usuarios que comparten gustos son activos, y el sistema detecta patrones en común entre ellos, cuando entre un nuevo usuario al sistema, si este detecta que tiene relación en común con estos, el sistema le recomendará imágenes que con mayor probabilidad cumplirán los requisitos o gustos del usuario en mayor grado que una imagen estándar. Además, debido a que se tratarán de grupos de usuarios muy concretos, este sistema es mucho más útil que un simple buscador de información, el cual devolvería el mismo tipo de resultado a cualquiera, aunque los gustos e intereses de ambos sean completamente distintos. 

Sin embargo, este sistema no tiene porque limitarse a aquellos usuarios que presentan dificultades con la comunicación escrita. Puede ser un aliciente y un elemento diferenciador que podría incorporarse a cualquier aplicación de mensajería, aportando una gran experiencia de usuario. Cuánta gente, en un momento determinado durante una conversación, no ha pensado en una imagen que vendría perfecta para esa situación, y ha tenido que bien buscarla entre la gran cantidad de imágenes que tiene en su dispositivo, o recurrir a salir de la aplicación, abrir el navegador y tener que buscarla, descargarla y luego enviarla (quizás ya demasiado tarde), o simplemente ha renunciado a la idea por pura pereza. El sistema utilizado en esta aplicación podría trasladarse fácilmente a cualquier otra, e integrarse con su infraestructura.

También se ha optado por un modelo de computación en la nube, prescindiendo de servidores físicos junto con toda la infraestructura física y tecnológica que implican. De esta manera, se aprovechan todas las ventajas que ofrecen este tipo de sistemas. Para empezar, usando las versiones gratuitas de prueba de las diferentes tecnologías que se emplearán, las cuales son más que suficientes para el desarrollo y pruebas iniciales de este proyecto, se eliminarán los gastos que implicarían la adquisición y el funcionamiento (electricidad necesaria), además del trabajo posterior que implica en esfuerzo humano el mantenimiento de un sistema de este calibre. Este tipo de infraestructura, ofrece también la posibilidad de un escalado fácil y sencillo. De esta manera, si el número de usuarios creciese demasiado, se podría adaptar fácilmente el sistema de manera que este contase con mayor potencia tanto computacional como de memoria. El uso de estas tecnologías, además, aportan gran cantidad de herramientas tanto para el mantenimiento como para ayudar en el desarrollo y el despliegue del servicio. Estas herramientas permitirán ofrecer una mejor experiencia de usuario, debido a su robustez, seguridad y disponibilidad.


Como se ha mencionado anteriormente, este proyecto esta especialmente orientado para personas que presentan dificultades en la comunicación escrita. Existen para este tipo de personas algunos recursos que, aunque no suficientes, si son muy útiles. Estos recursos se denominan \ac{SAAC}, y están especialmente diseñados para ayudar en el propósito de la comunicación de estas personas. El portal Aragonés de la comunicación aumentativa y alternativa \footnote{http://arasaac.org/index.php}, define los \ac{SAAC} como "formas de expresión distintas al lenguaje hablado, que tienen como objetivo aumentar (aumentativos) y/o compensar (alternativos) las dificultades de comunicación y lenguaje de muchas personas con discapacidad". Se han utilizado recursos de este portal para aportar de una mayor funcionalidad al sistema. Estos recursos se adaptan a personas con edades y habilidades motrices, cognitivas y lingüísticas muy dispares. En concreto se han utilizado los pictogramas tanto en color como en blanco y negro. 

Es necesario puntualizar que se ha desarrollado el sistema orientándolo a su uso con el idioma inglés. Esto es debido a que algunos de los recursos y herramientas utilizadas funcionan únicamente en inglés, además de que la búsqueda de recursos como imágenes es más rica debido a que gran parte de la gente añade las etiquetas e información necesaria para poder buscar estas imágenes en inglés.

Finalmente, la plataforma móvil escogida para el desarrollo de la aplicación es Android. Esto se debe a que además de ser una de las plataformas más utilizadas, los dispositivos Android son más asequibles que aquellos de otras plataformas como iOS, existiendo en Android un abanico más amplio donde escoger, encontrando desde dispositivos de gama media/baja relativamente asequibles, hasta aquellos de gama alta comparables a aquellos de la plataforma iOS.

\section{Estructura del documento}

A continuación, se muestra la estructura y distribución de este documento, nombrando cada uno de los capítulos así como una breve descripción sobre lo que contiene cada uno de estos.
\begin{definitionlist}

\item[Capítulo \ref{chap:intro}:\nameref{chap:intro}]

Es el presente capítulo, y en el se presenta una breve introducción al proyecto, así como la estructura que compondrá este documento, indicando que se va a encontrar en cada uno de los capítulos.


\item[Capítulo \ref{chap:objetivos}:\nameref{chap:objetivos}]

En este capítulo se presenta el principal objetivo que pretende alcanzar este proyecto, así como los subobjetivos en los que se puede dividir este y que permitirán alcanzar su consecución.

\item[Capítulo \ref{chap:intro}:\nameref{chap:antecedentes}]

Se presentan los fundamentos teóricos sobre los que este proyecto esta basado. En concreto se presenta toda la información referente a los sistemas de recomendación necesaria para entender posteriormente el por qué y el cómo de su uso.

\item[Capítulo \ref{chap:metodo}:\nameref{chap:metodo}]

En este capítulo se expone la metodología de trabajo que se ha empleado, explicando dicha metodología y justificando su uso, además de presentar los primeros resultados de la aplicación de esta. También se indican las herramientas software y hardware que se han utilizado para la elaboración del proyecto.

\item[Capítulo \ref{chap:arquitectura}:\nameref{chap:arquitectura}]

Se muestra la arquitectura global del sistema, así como las diferentes fases que se han seguido aplicando la metodología escogida en el capítulo anterior. Se realiza también, una descripción en profundidad sobre el algoritmo para la recomendación de imágenes que se ha empleado, así como una justificación del por qué se ha escogido este algoritmo y de las técnicas propias de los sistemas de recomendación que se aplican.

\item[Capítulo \ref{chap:resultados}:\nameref{chap:resultados}]

En este capítulo se analizan los resultados obtenidos tras la aplicación de las diferentes fases de la metodología, mostrando un ejemplo de ejecución del sistema y analizándolo.

\item[Capítulo \ref{chap:conclusiones}:\nameref{chap:conclusiones}]

Este es el último capítulo, y en el se describen las conclusiones que se pueden extraer tras el desarrollo tanto del proyecto como de la documentación asociada, así como una valoración personal y las posibles ampliaciones que se han planteado para realizar en el futuro.

\end{definitionlist}


% Local Variables:
%  coding: utf-8
%  mode: latex
%  mode: flyspell
%  ispell-local-dictionary: "castellano8"
% End:
