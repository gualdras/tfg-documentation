\chapter{Objetivos}
\label{chap:objetivos}

\noindent
En este capítulo se establecerá el principal objetivo que se pretende alcanzar mediante este proyecto, desglosando este a su vez en objetivos específicos que será necesario alcanzar para su consecución.

\section{Objetivo general}

El objetivo principal que se pretende lograr en este Trabajo Fin de Grado es el desarrollo de un sistema de recomendación de imágenes, que integrado en una aplicación de mensajería, sea capaz de a partir de una entrada proporcionada por el usuario, sugerir aquellas imágenes que, en base a ciertos criterios que se especificará a continuación, representen mejor aquello que el usuario desea transmitir. Este sistema estará principalmente orientado para aquellas personas que por un motivo u otro presenten dificultades a la hora de lidiar con las aplicaciones clásicas de mensajería.

Las imágenes a recomendar serán principalmente pictogramas, que resultan mas intuitivos que las imágenes tradicionales a la hora de representar conceptos, especialmente para gran parte del público para el que se diseña esta aplicación. Sin embargo estos pictogramas a veces pueden no ser suficientes, o puede darse el caso de que simplemente no se disponga de un pictograma adecuado para esa entrada. En estos casos será necesario hacer uso de recursos que se encuentren en la red, donde será muy importante hacer uso de fuentes fiables de información.

Respecto al procedimiento a seguir por parte del sistema, lo primero que será necesario es preprocesar la entrada, de manera que la búsqueda de imágenes no quede limitada a la palabra o palabras empleadas por el usuario. A continuación, será necesario establecer que imágenes tienen mas posibilidades de representar aquello que el usuario pretendía transmitir. Para esto se emplearán varias técnicas utilizadas en los sistemas de recomendación. Será necesario establecer perfiles entre los diferentes usuarios de la aplicación. Estos perfiles permitirán relacionar a usuarios con gustos similares de manera que sea más fácil sugerir imágenes que ya han sido previamente seleccionadas por otro usuarios con un perfil similar. De esta manera el sistema deberá mejorar con la cantidad de usuarios y el uso que se haga de este.

Es necesario recalcar que el objetivo principal del proyecto es el de desarrollar un sistema de recomendación, y no una aplicación de mensajería, por lo que la mayor parte del esfuerzo se centrará en el desarrollo de este sistema, careciendo la aplicación de algunas de las funcionalidades que caracterizan a las populares aplicaciones de mensajería ya existentes en el mercado.


\section{Objetivos específicos}

El objetivo discutido anteriormente será abarcado mediante los siguientes objetivos específicos.

\subsection{Aplicación Android}
Será necesario desarrollar una aplicación de mensajería para el sistema operativo Android. Esta aplicación deberá presentar las características básicas comunes a toda aplicación de mensajería. como una lista con los contactos del usuario que tengan la aplicación. También contará con un chat que permita la comunicación tanto mediante texto como mediante imágenes, y que permita al usuario realizar la entrada de manera oral y escrita. Para el almacenamiento de mensajes intercambiados por el usuario con otros se dispondrá de una base de datos.

\subsection{Sistema de comunicaciones}
El sistema de comunicaciones proporcionará una estructura que permita la comunicación entre los diferentes dispositivos que hagan uso de la aplicación. Para ello contará con los siguientes componentes. Por un lado, una base de datos situada en un servidor en la red y que contendrá a todos los usuarios de la aplicación, permitiendo a otros descubrir a aquellos contactos que también disponen de la aplicación. Por otro lado se deberá establecer un mecanismo que se encargue de hacer llegar los mensajes que envía un determinado usuario de la aplicación a otro.  

\subsection{Gestión de las imágenes}
Respecto a las imágenes se  desarrollará un mecanismo que nos permita gestionarlas. Esto incluye tanto el almacenamiento de las imágenes y la información referente a estas, como un mecanismo para poder extraer información de dichas imágenes y por último un mecanismo para la búsqueda en otros proveedores cuando las imágenes propias no sean suficientes. Las fuentes desde las que extraer estas imágenes serán configurables, y se podrán añadir fuentes según sea conveniente.

\subsection{Sistema de Recomendación}
Este sistema será el encargado de recomendar las imágenes al usuario. Para empezar tiene que ser capaz de determinar aquellas imágenes de entre todas las posibles que más se pueda adecuar a las necesidades de los usuarios. Para ello el sistema creará un perfil para cada usuario y relacionará aquellos perfiles que sean similares de manera que pueda utilizar la información de estos a la hora de sugerir. Esta información será principalmente aquellas imágenes que con la misma entrada o similar, otro usuario con un perfil afín haya seleccionado. No solo se usará la información que se pueda extraer de otros usuarios sino información propia de las imágenes. De esta manera, a partir de una entrada, lo primero que este sistema hará consistirá en una expansión semántica de esta entrada de manera que la recomendación no se reduzca únicamente a las palabras concretas que ha introducido este, sino al significado de estas. Tras esto, se realizará una consulta de aquellas imágenes que siguiendo los criterios mencionados anteriormente mejor puedan satisfacer la necesidad del usuario.





% Local Variables:
%  coding: utf-8
%  mode: latex
%  mode: flyspell
%  ispell-local-dictionary: "castellano8"
% End:
