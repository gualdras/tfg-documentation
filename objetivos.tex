\chapter{Objetivos}
\label{chap:objetivos}

\noindent
\drop{E}{}n este capítulo se establecerá el principal objetivo que se pretende alcanzar mediante este proyecto, desglosando este a su vez en objetivos específicos que será necesario alcanzar para su consecución.

\section{Objetivo general}

El objetivo principal que se pretende lograr en este \ac{TFG} es el desarrollo de un sistema de recomendación de imágenes, que integrado en una aplicación de mensajería, sea capaz de a partir de una entrada proporcionada por el usuario, sugerir aquellas imágenes que, en base a ciertos criterios que se especificará a continuación, representen mejor aquello que el usuario desea transmitir. Este sistema estará principalmente orientado para aquellas personas que por un motivo u otro presenten dificultades a la hora de trabajar con las aplicaciones clásicas de mensajería.

Las imágenes a recomendar serán principalmente pictogramas, que resultan más intuitivos que las imágenes tradicionales a la hora de representar conceptos, especialmente para gran parte del público para el que se diseña esta aplicación. Sin embargo, estos pictogramas a veces pueden no ser suficientes, o puede darse el caso de que simplemente no se disponga de un pictograma adecuado para la entrada del usuario. En estos casos será necesario hacer uso de recursos que se encuentren en la red, donde será muy importante hacer uso de fuentes fiables de información.

Respecto al procedimiento a seguir por parte del sistema, lo primero que será necesario es preprocesar la entrada, de manera que la búsqueda de imágenes no quede limitada a la palabra o palabras empleadas por el usuario. A continuación, será necesario establecer qué imágenes tienen más posibilidades de representar aquello que el usuario pretendía transmitir. Será necesario establecer perfiles entre los diferentes usuarios de la aplicación. Estos perfiles permitirán relacionar a usuarios con gustos similares de manera que sea más fácil sugerir imágenes que ya han sido previamente seleccionadas por otro usuarios con un perfil similar. De esta manera, el sistema deberá mejorar con la cantidad de usuarios y el uso que se haga de este.

Es necesario incidir en que el objetivo principal del proyecto es el de desarrollar un sistema de recomendación, y no una aplicación de mensajería, por lo que la mayor parte del esfuerzo se centrará en el desarrollo de este sistema, careciendo la aplicación de algunas de las funcionalidades que caracterizan a las populares aplicaciones de mensajería ya existentes en el mercado.


\section{Objetivos específicos}

El objetivo discutido anteriormente puede ser desglosado en los siguientes objetivos específicos.

\subsection{Desarrollo de una interfaz para la comunicación entre los usuarios}

Será necesario desarrollar una aplicación de mensajería. Esta aplicación deberá presentar las características básicas comunes a toda aplicación de mensajería, como una lista con los contactos del usuario que tengan la aplicación. También, contará con un chat que permita la comunicación tanto mediante texto como mediante imágenes, y que permita al usuario realizar la entrada de manera oral y/o escrita. Para el almacenamiento de mensajes intercambiados por el usuario con otros se dispondrá de una base de datos.

\subsection{Diseñar un protocolo de comunicación entre dispositivos}
Se desarrollará un protocolo de comunicaciones, el cual proporcionará una estructura que permita la comunicación entre los diferentes dispositivos y usuarios que hagan uso de la aplicación. Para ello contará con los siguientes componentes. Por un lado, un almacén de datos situado en la red, el cual contendrá a todos los usuarios de la aplicación junto con los datos de estos, permitiendo a los distintos usuarios descubrir a aquellos de entre sus contactos que también disponen de la aplicación. Por otro lado, se deberá establecer un mecanismo que se encargue de hacer llegar los mensajes que envía un determinado usuario de la aplicación a otro permitiendo la comunicación entre estos.  

\subsection{Implementar un repositorio flexible de imágenes}
Será necesario desarrollar un sistema capaz de almacenar y gestionar la información relativa a estas imágenes. Deberá permitir la posibilidad de añadir nuevas imágenes junto con información relativa de estas, además de poder modificar información de imágenes ya almacenadas.

\subsection{Desarrollo de un mecanismo para la obtención y análisis de nuevas imágenes}
Se deberá diseñar un mecanismo capaz de realizar búsquedas de imágenes en sitios de terceros cuando las imágenes propias sean insuficientes. Debe existir la posibilidad de poder configurar los sitios de terceros sobre los que se realizan esas búsquedas. También será necesario poder realizar un análisis de estas nuevas imágenes, con el fin de extraer la máxima información posible de estas.

\subsection{Diseño de un algoritmo matemático para la recomendación de imágenes}
Se deberán determinar aquellas variables que resultan más determinantes a la hora de recomendar las imágenes. Para ello se valorará si estas variables dependen de características propias de las imágenes únicamente, o si también influyen en ellas características propias de los usuarios. Con todos estos datos será necesario establecer un algoritmo que, valorando todos estos datos en el grado adecuado, sea capaz de determinar qué imágenes podrán satisfacer en mayor grado las necesidades del usuario y proceder así a su recomendación.




% Local Variables:
%  coding: utf-8
%  mode: latex
%  mode: flyspell
%  ispell-local-dictionary: "castellano8"
% End:
