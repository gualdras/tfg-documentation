\chapter{Conclusiones y trabajos futuros}
\label{chap:conclusiones}

\noindent
\drop{P}{}ara terminar, es necesario indicar que se ha cumplido con el objetivo de desarrollar una aplicación de mensajería capaz de recomendar imágenes a los usuarios a partir de sus gustos. Pese a núcleo principal que se pretendía conseguir era el sistema de recomendación, siendo la aplicación una mera herramienta para poder interactuar con este, se ha establecido una estructura sobre la que se podrá trabajar en un futuro. Existe un sistema de comunicaciones muy confiable, así como una muy buena base respecto a la aplicación de mensajería sobre la que en un futuro se podría trabajar para ampliar. Además, se ha desarrollado el sistema para almacenar una gran cantidad de información útil que podrá ser usada en el futuro para combinar con otras técnicas y poder mejorar aún más el sistema.

Como se puede comprobar en la mayoría de los portales webs en los que existen una gran cantidad de elementos, los sistemas de recomendación, pese a ser relativamente nuevos, tienen mucho futuro por delante. En este proyecto se ha conseguido desarrollar e integrar uno de estos sistemas en una aplicación real, en un tipo de aplicación en el que no es común, pero en el que tiene mucho potencial. Por tanto, este tipo de sistemas es uno de los que mayor futuro presentan en el ámbito de la computación

Alguno de los objetivos iniciales que se propusieron cuando se comenzó el trabajo no se han podido llevar a cabo debido a la magnitud de este trabajo. Destaca entre ellos la prueba una gran cantidad de usuarios reales para poder determinar el verdadero potencial y funcionamiento de este sistema. Otro objetivo que se planteó, pero que resultaba menos interesante que el anterior es la opción de incluir gifs además de imágenes. Como se verá más adelante, estas son algunas de las propuestas para posibles ampliaciones futuras del sistema.


En el plano personal, este proyecto me ha permitido aprender y mejorar en el empleo de un gran número de herramientas y tecnologías, así como de adquirir nuevos marcos teóricos muy útiles para la rama en la que quiero (y de hecho ya he empezado) especializarme. Entre los aptitudes tecnológicas, destacan la programación para la plataforma Android, y el aprendizaje sobre el funcionamiento de herramientas para el trabajo en la nube, como Google Cloud Platform y el resto de tecnologías asociadas a esta. También me ha parecido muy útil e interesante la base de datos Wordnet, que permite un gran abanico de posibilidades a la hora de tratar con lenguaje natural. Además, este trabajo me ha permitido mejorar en aquellas destrezas que en menor medida ya poseía como en el diseño y desarrollo de software, o la gestión y planificación.

En el contexto teórico, he aprendido mucho sobre los sistemas de recomendación, sistemas que creo que tienen mucha utilidad en un gran número de campos, y que al no tener mucho tiempo de vida, suponen un campo muy interesante sobre el que poder investigar y especializarse. También me ha mostrado como numerosos algoritmos o técnicas pueden combinarse entre sí para crear herramientas y sistemas más potentes.

\section{Trabajos futuros}

Debido al carácter de uno de los componentes de este sistema que es la aplicación de mensajería, existen en este sentido una cantidad de ampliaciones y mejoras prácticamente infinita, como mejoras en la interfaz, adición de nuevas funcionalidades como posibilidades de creación de grupos, configuración de perfil, listas de difusión... Sin embargo, el propósito de este proyecto nunca ha sido ser la competencia de Whatsapp o Telegram, si no el desarrollo de un sistema de recomendación integrado en una aplicación de estas características. Es por eso que se hará especial énfasis en las mejoras y ampliaciones que se pueden añadir para mejorar este proceso de recomendación, además de mejorar la experiencia de comunicación de los usuarios.


\begin{itemize}
\item Se plantea la posibilidad de trabajar con otro tipo de archivos multimedia como pueden ser gifs, pequeños videos o incluso efectos de sonido, los cuales aumentarían las posibilidades de la comunicación mediante medios no escritos. Por supuesto, se utilizarían mecanismos propios de los sistemas de recomendación para proporcionar estos recursos a los usuarios.
\item Una de las disciplinas de la computación que mayor aplicación tienen combinadas con los sistemas de recomendación es el aprendizaje automático. Esto se debe a que utilizando técnicas de aprendizaje automático es posible inferir relaciones existentes entre los usuarios que posteriormente pueden resultar de gran utilidad en los algoritmos colaborativos. También podría ser utilizado en la clasificación de los perfiles de cada usuario estableciendo categorías en los que estos están interesados, permitiendo aplicar filtros a las búsquedas. Es esta, de manera personal la vía de expansión más atractiva junto con el procesamiento de la entrada que se mostrará a continuación.
\item El aumento de sitios de los que se extraer recursos. Esto hace referencia tanto a incluir nuevos sitios web como ha crear algún tipo de portal en los que la gente puede subir nuevas imágenes o recursos multimedia y etiquetarlos así como permitir que otros usuarios puedan valorar estos. Además sería útil obtener nuevas formas de extraer información de estas las imágenes, ya sea de manera automática como permitiendo que los usuarios puedan etiquetarlas.
\item La mejora del procesamiento de la entrada es una de las posibles vías de mejora que mayor utilidad aportarían al sistema. Se podría desarrollar un procesador de lenguajes, que fuese capaz de analizar una frase determinando tanto la verdadera forma gramatical de una palabra como su significado léxico, de manera que la extracción de sinónimos y palabras relacionadas fuese siempre correcta. Sin embargo esta no es una tarea sencilla por la naturaleza del lenguaje natural.
\end{itemize}












































