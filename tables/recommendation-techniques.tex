


\begin{tabular}{p{.15\textwidth}p{.25\textwidth}p{.25\textwidth}p{.25\textwidth}}
  \tabheadformat
  \tabhead{Técnica}   &
  \tabhead{Información previa}      &
  \tabhead{Entrada} &
  \tabhead{Algoritmo}  \\
\hline
Colaborativo & Evaluaciones de \textbf{U} de los elementos en \textbf{I}. & Evaluación de \textbf{u} de los elementos en \textbf{I}. & Identifica a los usuarios en \textbf{U} con perfiles similares a \textbf{u}, y extrapola sus valoraciones de \textbf{i}.\\
\hline
Basada en contenido  & Características de los elementos en \textbf{I}. & Las valoraciones de \textbf{u} de los elementos en \textbf{I}. & Generar un clasificador que adapte el comportamiento respecto a las valoracioes de \textbf{u} y lo use en \textbf{i}. \\
\hline
Demográfica & Información demográfica sobre \textbf{U} y sus evaluaciones de los elementos en \textbf{I}. & Información demográfica sobre \textbf{u}. & Identificar aquellos usuarios que son demográficamente similares a \textbf{u} y extrapolar sus evaluaciones de \textbf{i}. \\
\hline
Basada en utilidad & Características de los elementos en \textbf{I}. & Una función de utilidad sobre los elementos en \textbf{I} que describa las preferencias de \textbf{u}. & Aplicar la función sobre los elementos y determinar la clasificación de \textbf{i}. \\
\hline
Basada en conocimiento & Características de los elementos en \textbf{I}. Conocimiento de como estos elementos cumplen las necesidades del usuario. & Una descripción de las necesidades o intereses de \textbf{u} & Inferir la afinidad del item \textbf{i} con las necesidades de \textbf{u}.\\
\hline
\end{tabular}


% Local variables:
%   coding: utf-8
%   ispell-local-dictionary: "castellano8"
%   TeX-master: "main.tex"
% End:
